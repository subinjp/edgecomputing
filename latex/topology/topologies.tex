\documentclass{beamer}

%
% Choose how your presentation looks.
%
% For more themes, color themes and font themes, see:
% http://deic.uab.es/~iblanes/beamer_gallery/index_by_theme.html
%

\mode<presentation>
{
  \usetheme{default}      % or try Darmstadt, Madrid, Warsaw, ...
  \usecolortheme{default} % or try albatross, beaver, crane, ...
  \usefonttheme{default}  % or try serif, structurebold, ...
  \setbeamertemplate{navigation symbols}{}
  \setbeamertemplate{caption}[numbered]
} 

\usepackage[english]{babel}
\usepackage[utf8x]{inputenc}
\usepackage{graphicx}
\graphicspath{ {figures/} }
\usepackage{siunitx}
\DeclareSIUnit[per-mode=symbol,per-symbol=p]{\MBps}{\mega\byte\per\second}
\DeclareSIUnit\giga{G}
\DeclareSIUnit\byte{B}
\DeclareSIUnit\bit{bit}

\sisetup{per-mode=symbol}

\usepackage{bookmark}

\title[Your Short Title]{Performance comparison of different topologies}
\author{Subin Joseph}
\institute{TU Kaiserslautern}
\date{27/09/2016}

\begin{document}

\begin{frame}
  \titlepage

\end{frame}

% Uncomment these lines for an automatically generated outline.
%\begin{frame}{Outline}
%  \tableofcontents
%\end{frame}

\section{Introduction}

\begin{frame}{Introduction}



\begin{itemize}
  \item Focus on the reliability comparison of different topologies through a quantitative study
  \item Compare the data latency,throughput and packet loss of following topolgies
  \begin{itemize}
  	\item Edge computing topology
  	\item Cloud computing topology
  	\item Edge plus cloud computing topology
  \end{itemize}

\end{itemize}


\end{frame}
\section{System Specification}
\begin{frame}{System Specification}
\begin{itemize}
  \item Used following system to test and evaluate the given task 
  \begin{itemize}
  	\item Linux System
  	\item Memory:\SI{15.5}{\giga\byte}
  	\item Processor: Intel CoreTM i3-6100 CPU @ 3.70GHz × 4 
  	\item OS Type :\SI{64}{\bit}

  \end{itemize}
\item Software Specification
  \begin{itemize}
  	\item NS3 Network Simulator
  	\item Wireshark-Packet Analyser
  	\item Eclipse IDE
  \end{itemize}
\end{itemize}
\end{frame}
\section{Configuration of three topologies}
\begin{frame}{Configuration of three topologies}
\begin{itemize}
	\item Used csma channel in between the WI-FI access points\big(Network address:192.168.0.x\big)
	\item Used point to point connection between WI-FI AP and dedicated servers\big(Network address:172.16.x.x\big) and wireless connection between wifi station points and wifi ap\big(Network address:10.0.x.x\big)
	\item Used UDP stream 
	\item Csma channel
	\begin{itemize}
		\item Data Rate:\SI{1000}Mbps
		\item Channel Delay:\SI{65600}{\nano\second}
	\end{itemize}
	\item P2P channel
	\begin{itemize}
		\item Data Rate:\SI{1000}Mbps
		\item Channel Delay:\SI{25000}{\nano\second}
	\end{itemize}
	\item Wireless channel\big(802.11ac\big)
	\begin{itemize}
		\item Data Rate:\SI{1040}Mbps
	\end{itemize}
	
\end{itemize}
\end{frame}
\section{Edge Computing Topology}

\begin{frame}{Edge Computing Topology}
\begin{figure}
\includegraphics[width=7cm, height=4cm]{edgecomp}
\centering
\end{figure}

\begin{itemize} 
	\item Stations belong to a wireless network communicate to the corresponding local server attached near to the wireless access 
\end{itemize}

\end{frame}

\section{Edge Cloud Topology}

\begin{frame}{Edge Cloud Topology}
\begin{figure}
\includegraphics[width=7cm, height=4cm]{edgecloud_comp}
\centering
\end{figure}

\begin{itemize}
	\item Here stations belong to two different wireless networks share a common cloud
	
\end{itemize}
\end{frame}

\section{Cloud Computing Topology}

\begin{frame}{Cloud Computing Topology}
\begin{figure}
\includegraphics[width=7cm, height=4cm]{cloud_comp}
\centering
\end{figure}

\begin{itemize}
	\item Here all stations belong to different wireless networks share a common cloud

\end{itemize}
\end{frame}
\section{Experiment and Results}
\begin{frame}{Experiment and Results}

\begin{itemize}
	\item Sent the UDP stream at the rate of \SI{10}Mbps from each station to the corresponding local servers and server sent back the stream at the same data rate to stations
	\item Measured the latency,throughput
	\item Latency: Difference between the time at which source send the packet and received the packet
	\item Time duration:\SI{10}{\second}
\end{itemize}
\end{frame}
\section{Experiment and Results}
\begin{frame}{Experiment and Results\big(Edge Computing\big)}
\begin{figure}
\includegraphics[width=8cm, height=6cm]{edgecomp_26}
\centering
\end{figure}
\begin{itemize}
	\item maximum latency experienced is \SI{.1182}{\second}
	\item minimum latency experienced is \SI{.00026}{\second} 
\end{itemize}

\end{frame}
\section{Experiment and Results}
\begin{frame}{Experiment and Results\big(Edge Computing\big)}
\begin{figure}
\includegraphics[width=8cm, height=6cm]{edgecompthrough}
\centering
\end{figure}
\begin{itemize}

	\item maximum throughput achieved is \SI{10.59}Mbps
	\item minimum throughput achieved is \SI{10.47}Mbps
	\item Average throughput achieved is \SI{10.48}Mbps
\end{itemize}

\end{frame}

\section{Experiment and Results}
\begin{frame}{Experiment and Results\big(Edge Cloud\big)}
\begin{figure}
\includegraphics[width=8cm, height=6cm]{edgecloudlog26}
\centering
\end{figure}
\begin{itemize}
	\item maximum latency experienced is \SI{1.854}{\second}
	\item minimum latency experienced is \SI{.00043}{\second} 
\end{itemize}

\end{frame}
\section{Experiment and Results}
\begin{frame}{Experiment and Results\big(Edge Cloud\big)}
\begin{figure}
\includegraphics[width=8cm, height=6cm]{edgecloud}
\centering
\end{figure}
\begin{itemize}

	\item maximum throughput achieved is \SI{10.50}Mbps
	\item minimum throughput achieved is \SI{1.78}Mbps
	\item Average throughput achieved is \SI{6.51}Mbps
\end{itemize}

\end{frame}
\section{Experiment and Results}
\begin{frame}{Experiment and Results\big(Cloud Computing\big)}
\begin{figure}
\includegraphics[width=8cm, height=6cm]{newcloudlatency}
\centering
\end{figure}
\begin{itemize}
	\item maximum latency experienced is \SI{2.145}{\second}
	\item minimum latency experienced is \SI{.0237}{\second} 
\end{itemize}

\end{frame}
\section{Experiment and Results}
\begin{frame}{Experiment and Results\big(Cloud Computing\big)}
\begin{figure}
\includegraphics[width=8cm, height=6cm]{cloudth}
\centering
\end{figure}
\begin{itemize}

	\item maximum throughput achieved is \SI{4.87}Mbps
	\item minimum throughput achieved is \SI{0.721}Mbps
	\item Average throughput achieved is \SI{1.80}Mbps
\end{itemize}

\end{frame}
\section{Experiment and Results}
\begin{frame}{Experiment and Results}
\begin{table}
\centering
\caption{Comparison of different attributes}
\label{tab:table1}
\begin{tabular}{ |p{2.85cm}|p{2.5cm}|p{2cm}|p{2.5cm}|  }
 \hline
  Parameters & Edge \newline computing & Edge \newline Cloud & Cloud\newline Computing\\ 
 \hline
 Max Packets \newline sent \&\ received& 39966 & 39982 & 15598\\
 \hline
 Min Packets \newline sent \&\ received & 39868 & 9973 & 2746\\
 \hline
 Avg Packets \newline sent \&\ received & 39911 & 28703 & 6391 \\ 
 \hline
 Max \newline Throughput\big(Mbps\big) & 10.59 & 10.50 & 4.87\\
 \hline 
 Minimum \newline Throughput\big(Mbps\big) & 10.47 & 1.78 & 0.721 \\
 \hline
Average \newline Throughput\big(Mbps\big) & 10.48 & 6.51 & 1.80 \\
 \hline
\end{tabular}
\end{table}

\end{frame}
\section{Conclusion}

\begin{frame}{Conclusion}
\begin{itemize}
	\item As expected,edge computing topology showed better results because of the dedicated servers for each wireless network
	\item Most of the stations in the cloud computing topology experienced high latency,less throughput 
\end{itemize}
\end{frame}
\end{document}


